\chapter{Introdução}

O objectivo deste livro é proporcionar uma instrução clara e uma
reflexão sobre meditação budista, tal como é ensinada por Ajahn Sumedho,
um bhikkhu (monge) da tradição Theravada. Os capítulos seguintes foram
editados a partir de palestras mais longas proferidas por Ajahn Sumedho
a meditadores, como uma abordagem prática à sabedoria do Budismo. Esta
sabedoria é também conhecida como Dhamma.

Somos convidados a usar este livro como um manual, passo-a-passo.
O primeiro capítulo tenta apresentar uma visão geral da
prática do Budismo e as secções subsequentes podem ser entendidas uma a
uma e seguidas de um período de meditação. O terceiro capítulo é uma
reflexão sobre a compreensão que se desenvolve através da meditação. O
livro termina com a forma de tomar os Refúgios e os Preceitos o que
redimensiona a prática da meditação dentro do contexto mais abrangente
do trabalho a ter com a mente. Estes podem ser pedidos formalmente aos
monásticos budistas (Saṅgha) ou serem assumidos pessoalmente. Eles
constituem os pressupostos através dos quais os valores espirituais são
trazidos ao mundo.

A primeira edição deste livro, em inglês, (2.000 cópias) foi impressa em
1985 -- aquando da abertura do Centro Budista Amarāvatī -- tendo esta se
esgotado rapidamente. O livro foi muito apreciado e algumas pessoas
ofereceram-se para patrocinar uma nova impressão; desta forma fizemos
uma revisão mais meticulosa do que a anterior e adicionámos algum

design para melhorar o livro como um todo; à parte disso o texto
manteve-se. Como este livro foi inteiramente produzido através de
contribuições voluntárias e de serviço ao Dhamma, é pedido aos leitores
que respeitem esta oferenda, mantendo-a disponível de forma gratuita.

Possam todos os seres realizar a Verdade.

\bigskip

{\raggedleft
Venerável Succito\\
Centro Budista Amarāvatī\\
Maio 1986
\par}


