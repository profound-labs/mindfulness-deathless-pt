\part{Investigação}

\chapter{O que é Meditação?}

A palavra meditação é uma palavra bastante usada nos dias de hoje,
abrangendo um vasto leque de práticas. No Budismo fala"-se de dois tipos
de meditação -- `\emph{samatha}' e `\emph{vipassanā}'. A meditação
`\emph{samatha}' é um tipo de meditação na qual se concentra a mente num
objecto não a deixando perambular por outras coisas. Escolhe"-se um
objecto de meditação, tal como a percepção da respiração, colocando"-se
toda a atenção nas sensações da inalação e da exalação. Através desta
prática começamos eventualmente a experienciar uma mente calma e ficamos
tranquilos, pois estamos a eliminar todas as outras impressões que
captamos através dos sentidos.

Os objectos usados para acalmar a mente são tranquilizadores (nem valia
a pena dizer!). Se quiserem excitar a mente vão para um sítio excitante,
não vão para um mosteiro Budista, mas sim a uma discoteca!\ldots{} Excitação
é algo no qual é fácil concentrarmo"-nos. É uma vibração tão forte que
nos atrai imediatamente. Se formos ao cinema e o filme for realmente
excitante ficamos fixados no ecrã. Não temos que fazer qualquer esforço
para observar algo que é excitante, romântico ou repleto de aventuras.
Mas, não estando acostumados, observar um objecto tranquilizador pode
ser tremendamente enfadonho. Quando estamos habituados a coisas muito
mais excitantes o que é que pode ser mais aborrecido do que ficarmos a
observar a nossa própria respiração? Para o
fazermos temos que `forçar' a nossa mente uma vez que a respiração não é
interessante, romântica, cheia de aventuras ou cintilante -- é
simplesmente o que é. Por isso temos de nos esforçar, pois não estamos a
receber qualquer estímulo vindo do exterior.

Neste tipo de meditação não tentamos criar qualquer imagem, apenas
concentramo"-nos na sensação normal do nosso corpo tal como ele está no
momento, mantendo a nossa atenção na respiração. Quando fazemos isto a
respiração torna"-se cada vez mais regulada e nós acalmamos\ldots{}
conheço pessoas que prescreveram meditação \emph{samatha} para a tensão
arterial elevada pois acalma o coração.

Assim, esta é a prática da tranquilidade. Podemos escolher diferentes
objectos para nos concentrarmos, mantendo a nossa atenção até sermos
absorvidos ou nos tornarmos um só com o objecto. Na realidade sente"-se
uma noção de unidade com o objecto no qual nos concentramos e isto é o
que é chamado de absorção.

A outra prática é `\emph{vipassanā}' ou meditação de \emph{insight}. Com
a meditação de \emph{insight} abrimos a mente para tudo. Não escolhemos
nenhum objecto em particular para nos concentrarmos ou absorvermos, mas
ficamos simplesmente a observar de forma a compreendermos as coisas como
elas são. E o que podemos observar ao ver ``as coisas como são'' é que
toda a experiência sensorial é impermanente. Tudo o que vemos, ouvimos,
cheiramos, saboreamos e tocamos, todas as condições mentais -- os nossos
sentimentos, memórias e pensamentos -- são condições da mente em
constante mudança, as quais surgem e cessam. Em \emph{vipassanā}
assumimos esta característica da impermanência (ou mudança) como uma
forma de olhar para toda a experiência sensorial que podemos observar
enquanto estamos aqui sentados.

Não se trata de uma mera atitude filosófica ou de uma crença numa teoria
budista específica. A impermanência é para ser totalmente reconhecida no
nosso interior através da abertura da mente à observação e
consciencialização da realidade. Não é uma questão de análise, assumindo
que a realidade deveria ser de uma determinada forma e, quando não é,
tentar perceber o porquê. Com a prática de \emph{insight} não estamos
nem a tentar analisar"-nos, nem a tentar mudar algo de maneira a que isso
se adeque aos nossos desejos. Nesta prática observamos simplesmente que
tudo aquilo que surge, seja mental ou físico, passa e desaparece.

Isto inclui os próprios órgãos dos sentidos, o objecto dos sentidos e a
consciência que surge aquando do contacto com esses objectos. Existem
também as condições mentais de gostar e de não gostar do que vemos,
cheiramos, saboreamos, sentimos ou tocamos; os nomes que atribuímos e as
ideias, palavras e conceitos que criamos à volta da experiência
sensorial. Grande parte da nossa vida é baseada em assunções feitas
devido a não se compreender e a não se investigar verdadeiramente a
realidade, tal qual como é. A vida, para quem não está desperto nem
consciente, pode tornar"-se deprimente ou confusa, especialmente quando
algo nos decepciona ou quando ocorre alguma tragédia. Nessas alturas
ficamos esmagados por não observamos a realidade do momento.

Na terminologia budista usamos a palavra Dhamma, ou Dharma, que
significa, `leis naturais'. Quando observamos e praticamos o Dhamma
abrimos a nossa mente para a realidade. Desta forma já não estamos a
reagir cegamente à experiência dos sentidos mas sim a compreendê"-la e,
através dessa compreensão, chegamos ao desapego. Começamos a
libertarmo"-nos de ser naturalmente esmagados, cegos e iludidos pelas
aparências. Estar consciente e desperto não é uma questão de nos
\emph{tornarmos} nisso, mas de o \emph{sermos}.

Observemos as coisas tal como elas são neste preciso momento ao invés de
fazermos algo agora para nos tornarmos conscientes no futuro. Observamos
o corpo como ele é, aqui sentado. Não pertence tudo à natureza? O corpo
humano pertence à terra, precisa de ser sustentado por coisas
provenientes da terra. Não podemos viver simplesmente do ar nem tentar
importar comida de Marte ou de Vénus. Temos que comer do que vive e
cresce nesta Terra. Quando o corpo morre regressa à terra, apodrece,
desfaz"-se e torna"-se novamente um com a terra. Segue as leis da
natureza, da criação e da destruição, do nascimento e da morte. Tudo o
que nasce não se mantém no mesmo estado: cresce, envelhece e morre. Tudo
na natureza, até o próprio universo, tem os seus períodos de existência,
nascimento e morte, começo e fim. Tudo aquilo de que nos apercebemos e
que concebemos é transitório, impermanente, e portanto nunca nos poderá
satisfazer de forma permanente.

Na prática do Dhamma também podemos observar o carácter insatisfatório
da experiência sensorial. Reparem simplesmente na vossa própria vida:
quando esperam satisfazer"-se através de objectos ou experiências
sensoriais, apenas o conseguem fazer temporariamente, talvez
gratificados e felizes nesse momento, mas imediatamente a seguir isso
muda. Isto acontece por não existir nada na consciência sensitiva que
tenha essência ou qualidade permanente, daí a experiência sensorial ser
uma constante mudança. Devido à ignorância e à falta de compreensão
dependemos imenso dessa experiência; habituamo"-nos a exigir, desejar e a
criar todo o tipo de coisas, apenas para de seguida nos sentirmos
terrivelmente desapontados, desesperados, pesarosos e assustados. Essas
mesmas expectativas e esperanças levam"-nos ao desespero, à angústia, à
lástima, à dor, ao lamento, à velhice, à doença e à morte.

Esta é a forma de examinar a consciência sensorial. A mente pode pensar
de forma abstracta, pode criar todo o tipo de ideias e imagens, pode
criar coisas muito apuradas ou grosseiras. Existe toda uma gama de
possibilidades desde estados muito aperfeiçoados de graça, felicidade e
êxtase até às mais densas e dolorosas misérias: do céu ao inferno,
usando uma terminologia mais pitoresca. Mas não existe nenhum Inferno
permanente nem nenhum Céu permanente. Na verdade não existe nenhum
estado permanente que possa ser concebido ou criado. Ao meditarmos,
assim que começamos a perceber as limitações, a qualidade insatisfatória
e a natureza transitória de toda a experiência sensorial, também
começamos a perceber que isto não é o eu ou o meu, é `\emph{anattā}',
`não"-eu'.

Quando tal é realizado começamos a libertar"-nos da identificação com as
condições sensoriais. Isto acontece não por as rejeitarmos, mas por as
compreendermos tal como elas são. Trata"-se de uma verdade a ser
realizada, não de uma crença. `\emph{Anattā}' não é uma crença budista
mas sim uma realização. Mas se não despendermos algum tempo da nossa
vida a tentar investigar e compreender, iremos provavelmente viver
sempre na convicção de que somos o nosso corpo. Podemos a dada altura
pensar `Oh, eu não sou o meu corpo' por termos lido alguma poesia
inspiradora ou uma nova abordagem filosófica, podemos até achar que é
uma boa ideia não se ser o corpo, mas ainda não realizámos isso. Alguns
intelectuais podem dizer `Nós não somos o corpo, o corpo não é o eu' --
dizê"-lo é fácil, mas sabê"-lo realmente é outra coisa. Através desta
prática de meditação, da investigação e compreensão da realidade,
começamos a libertarmo"-nos do apego. Quando já não tivermos expectativas
ou exigências então, obviamente, já não iremos sentir o desespero, a
pena e a angústia resultantes de quando não conseguimos aquilo que
queremos. Este é, de facto, o objectivo
-- `\emph{Nibbāna}' ou a realização de não nos agarrarmos a nenhum
fenómeno que tenha um princípio e um fim. Quando abrimos mão deste
habitual e insidioso apego a tudo quanto nasce e morre, começamos a
realizar a `não"-morte'.

Algumas pessoas vivem simplesmente reagindo à vida porque foram
condicionadas a fazê"-lo, como os cães de Pavlov. Se não estivermos
despertos para as coisas tal como elas são, então, na verdade, não somos
mais que uma mera criatura inteligente condicionada semelhante a um
ignorante cão condicionado. Podemos olhar com ar de superioridade para o
cão de Pavlov que saliva quando a campainha toca, mas reparem como
fazemos coisas tão semelhantes. Isto deve"-se ao facto de que na
experiência sensorial tudo é condicionante, não se trata de ser pessoa,
alma ou essência pessoal. Estes corpos, sentimentos, memórias e
pensamentos são percepções condicionadas na mente através da dor, devido
a termos nascido como seres humanos, nas nossas respectivas famílias,
classe social, raça, nacionalidade, etc.; dependendo se temos um corpo
masculino ou feminino, atraente ou não,\ldots{} e por aí adiante. Tudo
isto são apenas condições que não são nossas, que não somos nós. Estas
condições obedecem às leis da natureza, as leis naturais. Não podemos
dizer `Não quero que o meu corpo envelheça.' Bem, podemos dizê"-lo mas,
não importa o quão insistentes sejamos, o corpo continuará a envelhecer.
Não podemos esperar que o corpo nunca adoeça ou sinta dor ou ainda que
tenha sempre visão e audição perfeitas. Gostaríamos, não é verdade? `Eu
espero ser sempre saudável. Nunca serei um inválido e terei sempre uma
boa visão, nunca ficarei cego; tenho bom ouvido e por isso nunca serei
uma daquelas pessoas de idade para as quais os outros têm de gritar e
nunca ficarei senil e terei sempre controlo das minhas faculdades até
morrer aos noventa e cinco anos de idade, completamente desperto, com a
mente clara e alegre. E morrerei naturalmente enquanto durmo, sem dor.'
Isto é como todos gostaríamos que fosse. Até pode ser que, de entre nós,
alguns continuem a viver por bastante tempo e morram de forma bastante
edílica ou também pode acontecer que amanhã
os nossos olhos saltem das órbitas! Não é muito provável mas pode
acontecer! Contudo, o fardo da vida diminui consideravelmente quando
reflectimos nas limitações da nossa própria vida. Aí sabemos o que
podemos alcançar, o que podemos aprender. Tanta miséria humana resulta
de criarmos demasiadas expectativas e nunca sermos capazes de alcançar
tudo aquilo que desejamos.

Na nossa meditação e compreensão intuitiva das coisas tal como elas são,
vemos que beleza, refinamento e prazer são condições impermanentes --
assim como a dor, miséria e feiúra. Se conseguirmos realmente
compreender isso, poderemos desfrutar e aguentar o que quer que seja que
nos aconteça. Na verdade, grande parte da lição da vida consiste em
aprender a suportar aquilo que não gostamos em nós próprios e no mundo à
nossa volta; sermos capazes de ser pacientes e calmos, e não fazermos um
drama acerca das imperfeições da experiência sensorial. Podemo"-nos
adaptar, suportar e aceitar as características mutáveis do ciclo do
nascimento e da morte através do ``abrir mão'' e do desapego a esse
mesmo ciclo. Quando nos libertamos dessa identificação, experienciamos a
nossa verdadeira natureza, a qual é brilhante, clara, sábia, mas que já
não é algo pessoal, não é `eu' ou `meu' -- não existem metas nem apegos.
Apenas nos podemos apegar àquilo que não é a nossa consciência pura!

Os ensinamentos do Buddha são simples meios úteis, formas de observar a
experiência sensorial que nos ajudam a compreendê"-la. Não são
mandamentos ou dogmas religiosos nos quais temos de acreditar e aceitar.
São meras linhas guias que apontam para a maneira como as coisas são.
Desta forma não estamos a usar os ensinamentos do Buddha apegando"-nos a
eles como um fim em si próprio, mas apenas para nos relembrarmos de
estar despertos, alertas e conscientes de que tudo o que surge, cessa.

Trata"-se de uma observação e reflexão contínua e constante do mundo
sensitivo, visto este exercer uma influência forte e poderosa. Vivendo
num corpo assim, na sociedade actual, as pressões em todos nós são
incríveis. Tudo se move tão rapidamente -- a televisão e a tecnologia
desta era, os carros -- tudo tende a mover"-se a um ritmo muito rápido. É
tudo muito atraente, excitante e interessante, e tudo atrai os nossos
sentidos para o exterior. Simplesmente reparemos, quando vamos a
Londres, como toda a publicidade chama a nossa atenção para os cigarros
e garrafas de whisky! A nossa atenção é levada para coisas que podemos
comprar, reafirmando o renascer na experiência sensorial. A sociedade
materialista estimula a avidez de forma a gastarmos dinheiro e ainda
assim a nunca estarmos contentes com o que temos. Existe sempre algo
melhor, mais recente\ldots{} mais delicioso do que aquilo que ontem era
o mais delicioso\ldots{} e continua e continua, puxando"-nos
exteriormente para os objectos dos sentidos.

Mas na sala de meditação, não vimos aqui para olhar uns para os outros
ou para sermos levados ou atraídos para qualquer um dos objectos na
sala, mas para os usar de forma a relembrar. Somos relembrados quer a
concentrar as nossas mentes num objecto pacífico quer a abrir a mente a
investigar e a reflectir sobre a realidade. Cada um tem de experienciar
isto por si próprio. Não é a iluminação de alguém que vai iluminar os
restantes. Este é um movimento para o interior; não é o olhar para fora,
para alguém que é iluminado, que nos ilumina. Damos esta oportunidade de
encorajamento e encaminhamento para que aqueles que estão interessados o
possam realizar. Aqui podemos, a maior parte do tempo, ter a certeza de
que ninguém
nos vai roubar a mala! Hoje em dia não podemos confiar em
nada, mas o risco disso acontecer aqui é menor do que se tivéssemos
sentados em Piccadilly Circus. Os mosteiros Budistas são refúgios para
este tipo de abertura da mente. Esta é a nossa oportunidade enquanto
seres humanos. Como seres humanos temos uma mente que pode reflectir e
observar. Podemos observar, quer estejamos felizes ou miseráveis.
Podemos observar o ódio, o ciúme ou a confusão na nossa mente. Quando
estamos sentados e sentimo"-nos realmente confusos e aborrecidos existe
algo em nós que o sabe. Podemos odiar e reagir cegamente contra isso,
mas se formos mais pacientes conseguimos observar que isso é uma
condição temporária e transiente de confusão, ódio ou avareza. Mas um
animal não consegue fazer isso. Quando está zangado ele é somente isso,
perdidamente. Digam a um gato zangado para observar a sua raiva! Eu
nunca consegui ir muito longe com a nossa gata, ela não consegue
reflectir sobre a gula. Mas eu consigo e tenho a certeza de que todos
vocês também o conseguem fazer. Quando vemos comida deliciosa à nossa
frente o movimento da mente é o mesmo da nossa gata Doris. Mas nós
conseguimos observar a atracção animal para coisas que parecem boas e
que cheiram bem.

Quando observamos e compreendemos o impulso estamos a usar sabedoria.
Aquilo que observa a gula não é gula. A gula não se pode observar a ela
mesma, mas aquilo que não é gula pode observá"-la. Este observar é aquilo
a que chamamos `Buddha' ou
`Sabedoria de Buddha' -- a consciência das coisas tal como são na
realidade.
